\chapter{Revisão Bibliográfica}
\label{review}

A origem do CBM está intrinsecamente ligada ao problema de decisão da Propriedade da C1P. A C1P busca determinar se existe uma permutação das colunas de uma matriz binária tal que todos os 1's em cada linha formem um único bloco contíguo. O estudo formal desta propriedade e sua relação com a estrutura de matrizes foi estabelecido no trabalho seminal de Fulkerson e Gross \cite{fulkerson1965incidence}. Embora o artigo original tenha dado mais ênfase ao problema transposto — a permutação de linhas para garantir 1's consecutivos em colunas — e sua equivalência com grafos de intervalo, sua motivação fundamental era prática, vinda da análise da estrutura linear de genes. Este trabalho estabeleceu a base teórica para o CBM, que emerge como a sua generalização natural: enquanto a C1P é um problema de decisão, o CBM é um problema de otimização que visa encontrar a permutação que mais se aproxima dessa estrutura ideal quando a C1P não pode ser satisfeita.

\citeonline{fulkerson1965incidence} também demonstraram que o problema de decisão da C1P é solucionável em tempo polinomial. Posteriormente, o primeiro algoritmo com complexidade linear para sua resolução — o \textit{PQ-tree algorithm}, proposto por Booth e Lueker — foi apresentado em \citeonline{BOOTH1976335}. Uma revisão abrangente da literatura sobre problemas envolvendo a C1P e estruturas de blocos únicos em matrizes binárias foi mais tarde conduzida na tese \citeonline{dom2009recognition}.

\section{CBM}

\citeonline{kou1977polynomial} demonstrou que o CBM é um problema NP-difícil, e \citeonline{haddadi2002note} provou que essa propriedade se mantém mesmo quando restrito a matrizes (*, 2), nas quais cada linha contém um número arbitrário de 1’s e cada coluna possui exatamente dois 1’s. Apesar da importância prática do CBM, a literatura específica sobre o tema — especialmente no que diz respeito a métodos metaheurísticos — ainda é limitada.

\citeonline{HADDADI2015612} propôs um algoritmo de busca local que explora a vizinhança de uma solução a partir dos operadores de reinserção — remove uma coluna de sua posição inicial e a reinsere em outra posição da matriz — e de troca — permuta a posição de duas colunas na matriz —. O algoritmo realiza todos os movimentos possíveis na vizinhança e, caso identifique uma solução de melhor qualidade, reinicia o processo a partir dela. O trabalho também menciona experimentos computacionais cujos resultados sustentam a qualidade do método, entretanto, as instâncias ou os resultados obtidos não são descritos em mais detalhes.

Em sequência, \citeonline{SOARES2020104948} introduzem uma nova representação em grafos que permite abordar o CBM a partir das relações entre linhas da matriz binária, e não apenas das colunas, o que oferece uma nova perspectiva. Com base nessa representação, uma heurística construtiva gulosa e uma meta-heurística \textit{Iterated Local Search} (ILS) \cite{lourencco2003iterated} foram desenvolvidas, combinando estratégias de intensificação e diversificação. Além disso, o trabalho apresenta soluções ótimas obtidas por métodos exatos através da redução do CBM ao TSP, utilizando o Concorde. Os experimentos computacionais mostram que as abordagens propostas superam os melhores resultados anteriores \cite{HADDADI2015612}, alcançando \textit{gaps} de até -15.87\% e estabelecendo um novo estado da arte para o CBM.

\citeonline{HADDADI2021105273} apresentou um novo método, também baseado na ILS. O trabalho propôs operadores de busca local inéditos no âmbito do CBM até então, como o \textit{2 Best Insertion} (reinserção de duas colunas). A qualidade do método, entretanto, reside na redução do CBM ao TSP para a utilização do algoritmo de Lin-Kernighan (LK) \cite{helsgaun2000effective} na construção das soluções iniciais, o que conferem excelentes pontos de partida para a ILS. Outro ponto de destaque é a avaliação parcial em O(1) dos operadores da ILS, o que confere uma redução significativa nos tempos de execução. Por fim, o trabalho é concluído com a apresentação dos resultados obtidos através de experimentos que o afirmam como novo estado da arte para o CBM.

Em um trabalho subsequente, \citeonline{haddadi2022exponential} fez pequenos ajustes em relação à metodologia anterior. A principal diferença reside no processo de busca local: em vez de depender dos operadores com avaliação em O(1), a nova abordagem introduz uma Busca em Vizinhança Exponencial (\textit{Exponential Neighborhood Search} - ENS). A nova proposta é que a exploração dessa vizinhança é eficientemente reduzida à resolução de um pequeno TSP. Com isso, o algoritmo de Lin-Kernighan passa a ter uma função dupla: além de gerar a solução inicial de alta qualidade (como no método anterior), ele é agora o motor principal dentro da busca local, sendo usado heuristicamente para resolver o TSP correspondente à ENS em cada iteração. Essa integração mais profunda do LK, segundo os experimentos, alcançou soluções de melhor qualidade.

\section{Redução de CBM para TSP}

\section{PT}