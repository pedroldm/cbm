%% abtex2-modelo-trabalho-academico.tex, v-1.9.2 laurocesar
%% Copyright 2012-2014 by abnTeX2 group at http://abntex2.googlecode.com/ 
%%
%% This work may be distributed and/or modified under the
%% conditions of the LaTeX Project Public License, either version 1.3
%% of this license or (at your option) any later version.
%% The latest version of this license is in
%%   http://www.latex-project.org/lppl.txt
%% and version 1.3 or later is part of all distributions of LaTeX
%% version 2005/12/01 or later.
%%
%% This work has the LPPL maintenance status `maintained'.
%% 
%% The Current Maintainer of this work is the abnTeX2 team, led
%% by Lauro César Araujo. Further information are available on 
%% http://abntex2.googlecode.com/
%%
%% This work consists of the files abntex2-modelo-trabalho-academico.tex,
%% abntex2-modelo-include-comandos and abntex2-modelo-references.bib
%%

% ------------------------------------------------------------------------
% ------------------------------------------------------------------------
% abnTeX2: Modelo de Trabalho Academico (tese de doutorado, dissertacao de
% mestrado e trabalhos monograficos em geral) em conformidade com 
% ABNT NBR 14724:2011: Informacao e documentacao - Trabalhos academicos -
% Apresentacao
% ------------------------------------------------------------------------
% ------------------------------------------------------------------------

\documentclass[
	% -- opções da classe memoir --
	12pt,				% tamanho da fonte
	openany,			% capítulos começam em pág ímpar (insere página vazia caso preciso)
	oneside,			% para impressão em verso e anverso. Oposto a oneside
	a4paper,			% tamanho do papel. 
	% -- opções da classe abntex2 --
	%chapter=TITLE,		% títulos de capítulos convertidos em letras maiúsculas
	%section=TITLE,		% títulos de seções convertidos em letras maiúsculas
	%subsection=TITLE,	% títulos de subseções convertidos em letras maiúsculas
	%subsubsection=TITLE,% títulos de subsubseções convertidos em letras maiúsculas
	% -- opções do pacote babel --
	english,			% idioma adicional para hifenização
	french,				% idioma adicional para hifenização
	spanish,			% idioma adicional para hifenização
	brazil				% o último idioma é o principal do documento
	]{abntex2}

% ---
% Pacotes básicos 
% ---
\usepackage[T1]{fontenc}		% Selecao de codigos de fonte.
\usepackage[utf8]{inputenc}		% Codificacao do documento (conversão automática dos acentos)
\usepackage{lastpage}			% Usado pela Ficha catalográfica
\usepackage{indentfirst}		% Indenta o primeiro parágrafo de cada seção.
\usepackage{color}				% Controle das cores
\usepackage{graphicx}			% Inclusão de gráficos
\usepackage{microtype} 			% para melhorias de justificação
\usepackage{amsmath}
% ---
		
% ---
% Pacotes adicionais, usados apenas no âmbito do Modelo Canônico do abnteX2
% ---
\usepackage{lipsum}				% para geração de dummy text
% ---

% ---
% Pacotes de citações
% ---
\usepackage[brazilian,hyperpageref]{backref}	 % Paginas com as citações na bibl
\usepackage[alf]{abntex2cite}	% Citações padrão ABNT

% --- 
% CONFIGURAÇÕES DE PACOTES
% --- 

% ---
% Configurações do pacote backref
% Usado sem a opção hyperpageref de backref
\renewcommand{\backrefpagesname}{Citado na(s) página(s):~}
% Texto padrão antes do número das páginas
\renewcommand{\backref}{}
% Define os textos da citação
\renewcommand*{\backrefalt}[4]{
	\ifcase #1 %
		Nenhuma citação no texto.%
	\or
		Citado na página #2.%
	\else
		Citado #1 vezes nas páginas #2.%
	\fi}%
% ---

% ---
% Informações de dados para CAPA e FOLHA DE ROSTO
% ---
\titulo{Minimização de blocos consecutivos com Parallel Tempering e Lin-Kernighan-Helsgaun}
\autor{Pedro Lucas Damasceno}
\local{Ouro Preto}
\data{2025}
\orientador{Prof. Dr. Marco Antonio Moreira de Carvalho}
\instituicao{%
  UNIVERSIDADE FEDERAL DE OURO PRETO
  
  INSTITUTO DE CIÊNCIAS EXATAS E BIOLÓGICAS
  
  PROGRAMA DE PÓS-GRADUAÇÃO EM CIÊNCIA DA COMPUTAÇÃO}
\tipotrabalho{Tese (Mestrado)}
% O preambulo deve conter o tipo do trabalho, o objetivo, 
% o nome da instituição e a área de concentração 
\preambulo{Tese submetida ao Programa de Pós-Graduação em Ciência da Computação da Universidade Federal de Ouro Preto para obtenção do título de Mestre em Ciência da Computação.
}
% ---


% ---
% Configurações de aparência do PDF final

% alterando o aspecto da cor azul
\definecolor{blue}{RGB}{41,5,195}

% informações do PDF
\makeatletter
\hypersetup{
     	%pagebackref=true,
		pdftitle={\@title}, 
		pdfauthor={\@author},
    	pdfsubject={\imprimirpreambulo},
	    pdfcreator={LaTeX with abnTeX2},
		pdfkeywords={abnt}{latex}{abntex}{abntex2}{trabalho acadêmico}, 
		colorlinks=true,       		% false: boxed links; true: colored links
    	linkcolor=blue,          	% color of internal links
    	citecolor=blue,        		% color of links to bibliography
    	filecolor=magenta,      		% color of file links
		urlcolor=blue,
		bookmarksdepth=4
}
\makeatother
% --- 

% --- 
% Espaçamentos entre linhas e parágrafos 
% --- 

% O tamanho do parágrafo é dado por:
\setlength{\parindent}{1.3cm}

% Controle do espaçamento entre um parágrafo e outro:
\setlength{\parskip}{0.2cm}  % tente também \onelineskip

% ---
% compila o indice
% ---
\makeindex
% ---

% ----
% Início do documento
% ----
\begin{document}

% Retira espaço extra obsoleto entre as frases.
\frenchspacing

% ----------------------------------------------------------
% ELEMENTOS PRÉ-TEXTUAIS
% ----------------------------------------------------------
% \pretextual

% ---
% Capa
% ---
\imprimircapa
% ---

% ---
% Folha de rosto
% (o * indica que haverá a ficha bibliográfica)
% ---
\imprimirfolhaderosto*
% ---

% ---
% Inserir a ficha bibliografica
% ---

% Isto é um exemplo de Ficha Catalográfica, ou ``Dados internacionais de
% catalogação-na-publicação''. Você pode utilizar este modelo como referência. 
% Porém, provavelmente a biblioteca da sua universidade lhe fornecerá um PDF
% com a ficha catalográfica definitiva após a defesa do trabalho. Quando estiver
% com o documento, salve-o como PDF no diretório do seu projeto e substitua todo
% o conteúdo de implementação deste arquivo pelo comando abaixo:
%
% \begin{fichacatalografica}
%     \includepdf{fig_ficha_catalografica.pdf}
% \end{fichacatalografica}
% \begin{fichacatalografica}
% 	\vspace*{\fill}					% Posição vertical
% 	\hrule							% Linha horizontal
% 	\begin{center}					% Minipage Centralizado
% 	\begin{minipage}[c]{12.5cm}		% Largura
% 	
% 	\imprimirautor
% 	
% 	\hspace{0.5cm} \imprimirtitulo  / \imprimirautor. --
% 	\imprimirlocal, \imprimirdata-
% 	
% 	\hspace{0.5cm} \pageref{LastPage} p. : il. (algumas color.) ; 30 % cm.\\
% 	
% 	\hspace{0.5cm} \imprimirorientadorRotulo~\imprimirorientador\\
% 	
% 	\hspace{0.5cm}
% 	\parbox[t]{\textwidth}{\imprimirtipotrabalho~--% ~\imprimirinstituicao,
% 	\imprimirdata.}\\
% 	
% 	\hspace{0.5cm}
% 		1. Palavra-chave1.
% 		2. Palavra-chave2.
% 		I. Orientador.
% 		II. Universidade xxx.
% 		III. Faculdade de xxx.
% 		IV. Título\\ 			
% 	
% 	\hspace{8.75cm} CDU 02:141:005.7\\
% 	
% 	\end{minipage}
% 	\end{center}
% 	\hrule
% \end{fichacatalografica}
% ---

% ---
% Inserir errata
% ---
%\begin{errata}
%Elemento opcional da \citeonline[4.2.1.2]{NBR14724:2011}. Exemplo:
%
%\vspace{\onelineskip}
%
%FERRIGNO, C. R. A. \textbf{Tratamento de neoplasias ósseas %apendiculares com
%reimplantação de enxerto ósseo autólogo autoclavado associado ao plasma
%rico em plaquetas}: estudo crítico na cirurgia de preservação de %membro em
%cães. 2011. 128 f. Tese (Livre-Docência) - Faculdade de Medicina %Veterinária e
%Zootecnia, Universidade de São Paulo, São Paulo, 2011.
%
%\begin{table}[htb]
%\center
%\footnotesize
%\begin{tabular}{|p{1.4cm}|p{1cm}|p{3cm}|p{3cm}|}
%  \hline
%   \textbf{Folha} & \textbf{Linha}  & \textbf{Onde se lê}  & %\textbf{Leia-se}  \\
%    \hline
%    1 & 10 & auto-conclavo & autoconclavo\\
%   \hline
%\end{tabular}
%\end{table}
%
%\end{errata}
% ---

% ---
% Inserir folha de aprovação
% ---

% \begin{folhadeaprovacao}
% 
%   \begin{center}
%     {\ABNTEXchapterfont\large\imprimirautor}
% 
%     \vspace*{\fill}\vspace*{\fill}
%     \begin{center}
%       \ABNTEXchapterfont\bfseries\Large\imprimirtitulo
%     \end{center}
%     \vspace*{\fill}
%     
%     \hspace{.45\textwidth}
%     \begin{minipage}{.5\textwidth}
%         \imprimirpreambulo
%     \end{minipage}%
%     \vspace*{\fill}
%    \end{center}
%         
%    Trabalho aprovado. \imprimirlocal, 24 de novembro de 2012:
% 
%    \assinatura{\textbf{\imprimirorientador} \\ Orientador} 
%    \assinatura{\textbf{Professor} \\ Convidado 1}
%    \assinatura{\textbf{Professor} \\ Convidado 2}
%    %\assinatura{\textbf{Professor} \\ Convidado 3}
%    %\assinatura{\textbf{Professor} \\ Convidado 4}
%       
%    \begin{center}
%     \vspace*{0.5cm}
%     {\large\imprimirlocal}
%     \par
%     {\large\imprimirdata}
%     \vspace*{1cm}
%   \end{center}
%   
% \end{folhadeaprovacao}
% ---

% ---
% Dedicatória
% ---
%\begin{dedicatoria}
%   \vspace*{\fill}
%   \centering
%   \noindent
%   \textit{ Este trabalho é dedicado às crianças adultas que,\\
%   quando pequenas, sonharam em se tornar cientistas.} \vspace*{\fill}
%\end{dedicatoria}
% ---

% ---
% Agradecimentos
% ---
%\begin{agradecimentos}
%Os agradecimentos principais são direcionados à Gerald Weber, Miguel %Frasson,
%Leslie H. Watter, Bruno Parente Lima, Flávio de Vasconcellos Corrêa, %Otavio Real
%Salvador, Renato Machnievscz\footnote{Os nomes dos integrantes do %primeiro
%projeto abn\TeX\ foram extraídos de
%\url{http://codigolivre.org.br/projects/abntex/}} e todos aqueles que
%contribuíram para que a produção de trabalhos acadêmicos conforme
%as normas ABNT com \LaTeX\ fosse possível.
%
%Agradecimentos especiais são direcionados ao Centro de Pesquisa em %Arquitetura
%da Informação\footnote{\url{http://www.cpai.unb.br/}} da Universidade %de
%Brasília (CPAI), ao grupo de usuários
%\emph{latex-br}\footnote{\url{http://groups.google.com/group/latex-%br}} e aos
%novos voluntários do grupo
%\emph{\abnTeX}\footnote{\url{http://groups.google.com/group/abntex2} e
%\url{http://abntex2.googlecode.com/}}~que contribuíram e que ainda
%contribuirão para a evolução do \abnTeX.
%
%\end{agradecimentos}
% ---

% ---
% Epígrafe
% ---
\begin{epigrafe}
	\vspace*{\fill}
	\begin{flushright}
		\textit{A simplicidade é o último grau de sofisticação.\\
			(Leonardo da Vinci)}
	\end{flushright}
\end{epigrafe}
% ---

% ---
% RESUMOS
% ---

% resumo em português
\setlength{\absparsep}{18pt} % ajusta o espaçamento dos parágrafos do resumo
\begin{resumo}[Resumo]
	O problema de minimização de blocos consecutivos consiste em determinar uma permutação das colunas de uma matriz binária de modo a minimizar o número de blocos de 1’s consecutivos em cada linha. Esse problema apresenta aplicações em diversas áreas, como organização de arquivos, programação de linhas de produção, escalonamento e compressão de dados. Por se tratar de um problema de otimização combinatória NP-difícil, métodos aproximados têm se mostrado mais adequados para sua resolução. Neste contexto, este estudo descreve o desenvolvimento de um algoritmo baseado na aplicação da metaheurística \textit{Parallel Tempering} sobre soluções iniciais geradas pelo algoritmo Lin-Kernighan-Helsgaun, após a conversão do problema em um \textit{Traveling Salesman} Problem utilizando a matriz de distâncias de Hamming. A abordagem demonstrou eficiência ao produzir resultados equivalentes ou superiores ao estado da arte em todas as instâncias analisadas, sem que o tempo de execução se tornasse proibitivo.

	\textbf{Palavras-chave}: Metaheurística, Blocos Consecutivos, Parallel Tempering, Distâncias de Hamming, Lin-Kernighan-Helsgaun.
\end{resumo}

% resumo em inglês

\begin{resumo}[Abstract]
	\begin{otherlanguage*}{english}
		The problem of consecutive blocks minimization consists in determining a permutation of the columns of a binary matrix so as to minimize the number of consecutive 1’s in each row. This problem has applications in several areas, such as file organization, production line scheduling, task sequencing, and data compression. Since it is an NP-hard combinatorial optimization problem, approximate methods have proven to be more suitable for obtaining high-quality solutions. In this context, this study presents the development of an algorithm based on the application of the Parallel Tempering metaheuristic over initial solutions generated by the Lin-Kernighan-Helsgaun algorithm, after transforming the problem into a Traveling Salesman Problem using the Hamming distance matrix. The proposed approach proved effective, producing results equivalent or superior to the current state of the art for all analyzed instances, without incurring prohibitive computational times.

		\textbf{Keywords}: Metaheuristic, Consecutive Blocks, Parallel Tempering, Hamming Distances, Lin-Kernighan-Helsgaun.
	\end{otherlanguage*}
\end{resumo}
% ---

% ---
% inserir lista de ilustrações
% ---
\pdfbookmark[0]{\listfigurename}{lof}
\listoffigures*
\cleardoublepage
% ---

% ---
% inserir lista de tabelas
% ---
\pdfbookmark[0]{\listtablename}{lot}
\listoftables*
\cleardoublepage
% ---

% ---
% inserir lista de abreviaturas e siglas
% ---
\begin{siglas}
	\item[C1P]  Propriedade dos 1's Consecutivos (\textit{Consecutive Ones Property})
	\item[CBM]  Problema de Minimização de Blocos Consecutivos (\textit{Consecutive Blocks Minimization})
	\item[PT]   Revenimento Paralelo (\textit{Parallel Tempering})
    \item[TSP]  Problema do Caixeiro Viajange (\textit{Traveling Salesman Problem})
    \item[ILS]  Busca Local Iterada (\textit{Iterated Local Search})
    \item[LK]   Algoritmo de Lin-Kernighan
    \item[ENS] Busca em Vizinhança Exponencial (\textit{Exponential Neighborhood Search})
\end{siglas}
% ---

% ---
% inserir lista de símbolos
% ---
\begin{simbolos}
	\item[$ \Gamma $] Letra grega Gama
	\item[$ \Lambda $] Lambda
	\item[$ \zeta $] Letra grega minúscula zeta
	\item[$ \in $] Pertence
\end{simbolos}
% ---

% ---
% inserir o sumario
% ---
\pdfbookmark[0]{\contentsname}{toc}
\tableofcontents*
\cleardoublepage
% ---

\chapter{Introdução}
\label{introduction}

A análise de matrizes binárias (0, 1) é fundamental para a abstração e modelagem de problemas de otimização discreta e combinatória. Um problema canônico neste contexto, formalizado por \citeonline{fulkerson1965incidence}, é a verificação da Propriedade dos 1's Consecutivos (\textit{Consecutive Ones Property} - C1P). Esta propriedade é satisfeita se existir uma permutação das colunas da matriz que resulte em blocos contíguos de 1's em cada linha. A existência de tal permutação valida a consistência dos dados com um modelo estrutural perfeitamente linear, uma questão originalmente motivada pela análise da estrutura de genes. Para exemplificar os conceitos, considere a Matriz $A$~\eqref{eq:matriz_a} (C1P), e a Matriz $B$~\eqref{eq:matriz_b}, para a qual nenhuma permutação de colunas é capaz de satisfazer a propriedade.

\begin{figure}[h!] % Usa um ambiente 'figure' para manter as matrizes juntas
    \centering
    \begin{minipage}{0.4\textwidth}
        \centering
        \begin{equation} \label{eq:matriz_a}
            A = \begin{bmatrix}
                \textbf{1} & \textbf{1} & \textbf{1} & 0          \\
                0          & \textbf{1} & \textbf{1} & 0          \\
                0          & 0          & \textbf{1} & \textbf{1}
            \end{bmatrix}
        \end{equation}
    \end{minipage}
    \begin{minipage}{0.4\textwidth}
        \centering
        \begin{equation} \label{eq:matriz_b}
            B = \begin{bmatrix}
                0          & \textbf{1} & \textbf{1} & 0          \\
                0          & \textbf{1} & 0          & \textbf{1} \\
                \textbf{1} & 0          & \textbf{1} & \textbf{1}
            \end{bmatrix}
        \end{equation}
    \end{minipage}
\end{figure}

Na prática, a maioria das matrizes que representam problemas reais não possui a C1P. Esta limitação motiva a formulação de um problema de otimização mais geral: o Problema de Minimização de Blocos Consecutivos (\textit{Consecutive Blocks Minimization} - CBM). O objetivo do CBM é determinar a permutação de colunas que minimiza o número total de blocos de 1's na matriz, representando a melhor aproximação possível a uma estrutura linear ideal. A Matriz $B$ ~\eqref{eq:matriz_b} possui 5 blocos de 1's, entretanto, é possível permutar suas colunas de forma a diminuir essa quantidade. A Matriz $C$~\eqref{eq:matriz_c} apresenta uma permutação ótima das colunas da Matriz $B$~\eqref{eq:matriz_b}, reduzindo o total de blocos para 4.

\begin{figure}[h!] % Usa um ambiente 'figure' para manter as matrizes juntas
    \centering
    % Matriz C (Ótima)
    \begin{minipage}{0.4\textwidth}
        \centering
        \begin{equation} \label{eq:matriz_c}
            C =
            \begin{array}{c}
                % Índices das colunas (permutados)
                \begin{array}{cccc}
                    1 & 4 & 2 & 3
                \end{array} \\[3pt]
                % Matriz propriamente dita
                \left[
                    \begin{array}{cccc}
                        0          & 0          & \mathbf{1} & \mathbf{1} \\
                        0          & \mathbf{1} & \mathbf{1} & 0          \\
                        \mathbf{1} & \mathbf{1} & 0          & \mathbf{1}
                    \end{array}
                    \right]
            \end{array}
        \end{equation}
    \end{minipage}
\end{figure}

O CBM é um problema de relevância multidisciplinar e possui aplicações em áreas diversas, como arqueologia (\textit{Sequence Dating Problem}) \cite{kendall1969incidence}, genética computacional (\textit{Physical Mapping Problem}) \cite{alizadeh1995physical}, compressão de dados em larga escala \cite{lemire2011reordering}, e várias outras. Por se tratar de um problema NP-difícil \cite{kou1977polynomial}, métodos heurísticos têm se provado mais adequados à sua resolução.

Nesse contexto, este trabalho propõe um algoritmo híbrido para a resolução do CBM, combinando o Revenimento Paralelo (\textit{Parallel Tempering} — PT) com o algoritmo Lin-Kernighan-Helsgaun (LKH). A metodologia consiste em converter as instâncias do CBM em instâncias do Problema do Caixeiro Viajante (\textit{Traveling Salesman Problem} — TSP), utilizando a matriz de distâncias de Hamming, e empregar o LKH para construir um conjunto de soluções iniciais de alta qualidade. Essas soluções são utilizadas como ponto de partida para o PT, que, por apresentar resultados expressivos em problemas de permutação análogos \cite{ALMEIDA2025107000}, é utilizado como estratégia de diversificação para expandir o espaço de soluções explorado.

\section{Justificativa}

A relevância do CBM é evidenciada por sua vasta aplicabilidade em domínios práticos distintos, que incluem a seriação arqueológica, o mapeamento genético e a compressão de dados em larga escala \cite{kendall1969incidence, alizadeh1995physical, lemire2011reordering}.

Contudo, a natureza NP-difícil do problema \cite{kou1977polynomial} impõe uma complexidade computacional significativa. A obtenção de soluções ótimas por métodos exatos é, na maioria dos casos práticos, inviável, exigindo um tempo de execução que cresce exponencialmente com o tamanho da instância. Esta complexidade intrínseca é a principal motivação para o desenvolvimento e aprimoramento de métodos heurísticos e metaheurísticos, que buscam encontrar soluções de alta qualidade em tempo computacional razoável.

Embora diversas abordagens heurísticas tenham sido propostas para o CBM, a busca por métodos que avancem o estado da arte — ou seja, que encontrem soluções com um número de blocos ainda menor ou que o façam de forma mais eficiente — é um desafio contínuo. Este trabalho se justifica na exploração de um novo método que propõe a sinergia entre duas técnicas de vanguarda da otimização combinatória: uma das mais poderosas heurísticas para o problema do TSP simétrico e uma robusta metaheurística paralela de diversificação.

\section{Objetivos}

O principal objetivo deste trabalho é desenvolver um algoritmo híbrido, baseado na combinação dos métodos LKH e PT, capaz de avançar o estado da arte do CBM, identificando soluções de menor custo para as instâncias da literatura em tempo computacional viável. São objetivos específicos:

\begin{enumerate}
    \item Elaborar uma rigorosa revisão da literatura relacionada ao CBM e problemas permutacionais análogos a fim de compreender abordagens anteriores e identificar lacunas;
    \item Desenvolver um novo método baseado na combinação do algoritmo LKH e da metaheurística PT;
    \item Realizar experimentos computacionais para configurar o método desenvolvido, analisar suas propriedades a partir das estatísticas de execução e comparar os resultados obtidos com o atual estado da arte.
\end{enumerate}

\section{Organização do Trabalho}

Este trabalho está organizado como descrito a seguir. O Capítulo \ref{review} apresenta a revisão da literatura relevante em relação ao CBM. Em seguida, no Capítulo \ref{theory}, uma base teórica sólida é estabelecida, abrangendo a definição do problema, os objetivos específicos considerados neste trabalho e uma introdução às técnicas empregadas no seu desenvolvimento. O Capítulo \ref{methods} discute minuciosamente a implementação das técnicas incorporadas no algoritmo, enquanto o Capítulo \ref{experiments} apresenta os resultados obtidos pelo método desenvolvido, bem como uma comparação com o estado da arte. Por fim, o Capítulo \ref{conclusion} encerra este trabalho, fornecendo uma síntese dos resultados alcançados e delineando as próximas etapas de pesquisa.
\chapter{Revisão Bibliográfica}
\label{review}

\chapter{Fundamentação Teórica}
\label{theory}
\chapter{Desenvolvimento}
\label{methods}

\chapter{Experimentos Computacionais}
\label{experiments}
\chapter{Conclusão}
\label{conclusion}

\bibliography{referencias}

%---------------------------------------------------------------------
% INDICE REMISSIVO
%---------------------------------------------------------------------
\phantompart
\printindex
%---------------------------------------------------------------------

\end{document}